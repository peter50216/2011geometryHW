\documentclass[10pt,a4paper]{article}
\usepackage{amsmath, amssymb, amsthm}
%加這個就可以設定字體
\usepackage{fontspec}
\usepackage{xkeyval} %MikTeX 2.9 版本相容有誤, 以此修正
%使用xeCJK,其他的還有CJK或是xCJK
\usepackage{xeCJK}
%設定英文字型,不設的話就會使用預設的字型
%\setmainfont{Times New Roman}
%設定中英文的字型
%字型的設定可以使用系統內的字型,而不用像以前一樣另外安裝
\setCJKmainfont{文泉驛微米黑}
\setCJKfamilyfont{lh}{LiHei Pro}
\newcommand{\LiHei}{\CJKfamily{lh}}
%中文自動換行
\XeTeXlinebreaklocale "zh"
%文字的彈性間距
\XeTeXlinebreakskip = 0pt plus 1pt
%設定段落之間的距離
\setlength{\parskip}{0.3cm}
%設定行距
%\linespread{1.5}\selectfont

%先暫時用水泥字型,如果任何人看不下去就改吧XD
\usepackage[T1]{fontenc}
\usepackage{concmath}
%\usepackage{mathptmx} %Times

\newcounter{theProblemCounter}
\newtheorem{problem}[theProblemCounter]{Problem}

\begin{document}
\title{{\fontspec{Copperplate Gothic Bold}Geometry Homework 2}}
\author{{\it{B96201044}} {\LiHei 黃上恩}, {\it{B98901182}} {\LiHei 時丕勳}, {\it{K0020100x}} {\LiHei 劉士瑋}}
\date{\today}
\maketitle

%第三題
\setcounter{theProblemCounter}{2}
\begin{problem}[P47: 5]
If a closed plane curve $C$ is contained inside a disk of radius $r$, prove that there exists a point $p\in C$ such that the curvature $\kappa$ of $C$ at $p$ satifies $|\kappa|\ge 1/r$.
\end{problem}
\begin{proof}
Let $X(s)$ denote the curve $C$, where $s\in [0, l]$ is an arc-length parameter, that is, $\|X'(s)\|\equiv 1$. Since $C$ is contained inside a disk of radius $r$, let $A$ be the centre of the disk. So we have
\begin{equation} \|X(s)-A\|\le r\end{equation}
Consider $f(s)=\left\langle X(s)-A, X(s)-A\right\rangle$. Since $[0, l]$ is compact, the maximum exists, denoting by $f(s') = \max_{s\in[0,l]}f(s)$. Therefore, we have $f'(s')=0$ and $f''(s') \le 0$. Now 
\begin{equation}
	f''(s)=2\left(\|X'(s)\|^2+\kappa(s)\left\langle X(s)-A, N(s)\right\rangle\right),
\end{equation}
where $X''(s)=\kappa(s)N(s)$ and $N(s)$ is the normal vector. Take $s=s'$ in (2) we have $f''(s') \le 0$ and hence
\begin{equation}
\kappa(s')\left\langle X(s)-A, N(s)\right\rangle \le -1
\end{equation}
This implies
\begin{equation}
|\kappa(s')\left\langle X(s)-A, N(s)\right\rangle| \ge 1
\end{equation}
By (1), $|\left\langle X(s)-A, N(s)\right\rangle| \le \|X(s)-A\|\cdot \|N(s)\| \le r$. We have $|\kappa(s')|\ge 1/r$ as desired.
\end{proof}

%第四題 %我又打錯題目了= =
\setcounter{theProblemCounter}{3}
\begin{problem}[P23: 4, 僅討論平面情形]
Assume that all normals of a parame-trized curve pass through a fixed point. Prove that the trace of the curve is contained in a circle.
\end{problem}
\begin{proof}
Let $P$ be the fixed point, and let $X(s)$ be this curve. Then from description, $\left\langle X(s)-P, X'(s)\right\rangle\equiv 0$ for all $s$. Let $f(s) = \|X(s)-P\|^2$, then we have $f'(s) = 2\left\langle X(s)-P, X'(s)\right\rangle = 0$ for all $s$. This implies the trace of the curve is contained in a circle centered at point $P$ with radius $\sqrt{f(s_0)}$ for some $s_0$.
\end{proof}

%第五題
\setcounter{theProblemCounter}{4}
\begin{problem}
以 $t=0$ 開始將曲線 $(t^2, t^3)$ 化成長度參數。並討論 $t=0$ 時的曲率。
\end{problem}

\begin{proof}
Consider $t>0$, the length of the curve of $t$ is
\begin{equation}
\int_0^t 3t\sqrt{(4/9)+t^2}\ dt = \int_0^t \frac32\sqrt{(4/9)+t^2}\ dt^2 = \left(\frac49+t^2\right)^{3/2}-\frac{8}{27}
\end{equation}
Let $s>0$ be the arc-length parameter, note that $s>0$ equivalent to $t>0$, so we have $s=(4/9+t^2)^{3/2}-8/27$ and hence
\begin{equation}
t = \sqrt{\left(s+\frac{8}{27}\right)^{2/3} - \frac{4}{9}}
\end{equation}
Therefore the curve with arc-length parameter $s>0$ is
\begin{equation}
\left(\left(s+8/27)\right)^{2/3} - 4/9, 
\left(\left(s+8/27\right)^{2/3} - 4/9\right)^{3/2}\right)
\end{equation}
By symmetry, for the case $s<0$ the corresponding curve is
\begin{equation}
\left(\left(8/27-s)\right)^{2/3} - 4/9, 
-\left(\left(8/27-s\right)^{2/3} - 4/9\right)^{3/2}\right)
\end{equation}
We can write them together to get the result,
\begin{equation}
\left(\left(8/27+|s|)\right)^{2/3} - 4/9, 
sign(s)\cdot\left(\left(8/27+|s|\right)^{2/3} - 4/9\right)^{3/2}\right)
\end{equation}
For $t\ne 0$, $X'(t)=(2t, 3t^2)\ne 0$, so the curvature is
$\kappa(t)=((2t)(6t)-(3t^2)(2)) / \sqrt{4t^2+9t^4} = 6t / \sqrt{4+9t^2} \to 0$ as $t\to 0$. So when $t=0$, the curvature can be defined to be 0 so that $\kappa(t)$ at $t=0$ is continuous.
\end{proof}

%第六題
\setcounter{theProblemCounter}{5}
\begin{problem}
\begin{enumerate}
\item[]
\item[(a)] 以原點為中心,將 $y=f(x)$ 的圖形縮放 $\lambda$ 倍,並說明新的圖形是 $y=\lambda f(\frac{x}{\lambda})$ 的函數圖形。
\item[(b)] 討論曲率的變化。
\end{enumerate}
\end{problem}
\begin{proof}
\begin{enumerate}
\item[]
\item[(a)] 原本圖形上的點 $(x, f(x))$ 經過縮放後會到 $(\lambda x, \lambda f(x)) = (\lambda x, \lambda f(\frac{\lambda x}{\lambda}))$,所以新的函數圖形就是 $y = \lambda f(\frac{x}{\lambda})$。
\item[(b)] 原本的曲率是
\begin{equation}
\kappa = \frac{\left|\begin{array}{cc} x' & y' \\ x'' & y''\end{array}
\right|
}{(x'^2+y'^2)^{3/2}}
= \frac{\left|\begin{array}{cc} 1 & f' \\ 0 & f''\end{array}
\right|
}{(1+f'^2)^{3/2}}
= \frac{f''}{(1+f'^2)^{3/2}}
\end{equation}
新的曲率是
\begin{equation}
\kappa_{\mbox{new}} = \frac{\left|\begin{array}{cc} x' & y' \\ x'' & y''\end{array}
\right|
}{(x'^2+y'^2)^{3/2}}
= \frac{\left|\begin{array}{cc} 1 & \lambda f'\cdot \frac{1}{\lambda} \\ 0 & \frac{1}{\lambda}\cdot f''\end{array}
\right|
}{(1+f'^2)^{3/2}}
= \frac{\frac{1}{\lambda}f''}{(1+f'^2)^{3/2}}
\end{equation}
是原本的 $1/\lambda$ 倍。
\end{enumerate}
\end{proof}

%第七題
\setcounter{theProblemCounter}{6}
\begin{problem}
如圖,有一橢圓,其焦點為 $O_1$ 和 $O_2$,設 $L$ 切橢圓於 $P$,且 $L$ 與 $\overline{O_2P}$ 之夾角為 $\theta$。以 $\theta$ 為參數,說明曲率 $\kappa\propto\sin^3\theta$
\end{problem}
\begin{proof}
不妨假設 $O_1, O_2$ 皆落在 $X$ 軸上,我們將此橢圓參數化為 $(a\cos t, b\sin t)$,其中 $t\in [0, 2\pi]$ 而且 $a>b$。於是可得橢圓之兩焦點座標分別是 $(c, 0), (-c, 0)$ 其中 $c=\sqrt{a^2-b^2}$。計算此曲線的曲率為
\begin{equation}
\kappa(t) = \frac{\left|
\begin{array}{cc}
-a\sin t & b\cos t\\
-a\cos t & -b\sin t\end{array}
\right|}{(a^2 \sin^2 t + b^2 \cos^2 t)^{3/2}}
=\frac{ab}{(a^2 \sin^2 t + b^2 \cos^2 t)^{3/2}}
\end{equation}
現在來計算 $\sin\theta(t)$,其實 $\theta(t)$ 就是向量 $O_2P = (a\cos t-c, b\sin t)$ 與切向量 $(-a\sin t, b\cos t)$ 的有向夾角,所以
\begin{align}
\sin\theta(t) =& \frac{\left|
\begin{array}{cc}  a\cos t-c& b\sin t\\
-a\sin t & b\cos t\end{array}
\right|
}{\sqrt{(a\cos t-c)^2 + b^2\sin^2t} \sqrt{a^2\sin^2t+b^2\cos^2t}}\\
=& \frac{ab-bc\cos t}
{\sqrt{a^2\cos^2t-2ac\cos t+c^2 + b^2\sin^2t}\sqrt{a^2\sin^2t+b^2\cos^2t}}\\
=& \frac{b(a-c\cos t)}
{\sqrt{a^2\cos^2t-2ac\cos t+a^2-b^2\cos^2t}\sqrt{a^2\sin^2t+b^2\cos^2t}}\\
=&\frac{b(a-c\cos t)}
{\sqrt{c^2\cos^2t-2ac\cos t+a^2}\sqrt{a^2\sin^2t+b^2\cos^2t}}\\
=&\frac{b(a-c\cos t)}
{\sqrt{(a-c\cos t)^2}\sqrt{a^2\sin^2t+b^2\cos^2t}}\\
=&\frac{b}{\sqrt{a^2\sin^2t+b^2\cos^2t}}
\end{align}
而從 (17) 推到 (18) 是因為 $c\cos t \le c < a$。於是
\begin{equation}
\kappa(t) = \frac{ab}{(a^2\sin^2t+b^2\cos^2t)^{3/2}}
= \frac{a}{b^2} \frac{b^3}{(a^2\sin^2t+b^2\cos^2t)^{3/2}} \propto \sin^3\theta
\end{equation}
\end{proof}

%第九題
\setcounter{theProblemCounter}{8}
\begin{problem}
如圖,有 regular curve $\gamma(t)$,$\gamma_0=\gamma(0)$,$N_0=N(0)$,$L_0=\{\gamma_0+vN_0\}$。現考慮直線 $L_t=\{\gamma(t)+uN(t)\}$,令 $P(t)=L_t\cap L_0$ 證明
\[\kappa(0)\ne 0\Rightarrow \lim_{t\to 0}P(t)=\gamma_0 + \frac{1}{\kappa(0)}N_0\]
\end{problem}
\begin{proof}
Let $P(t)=\gamma_0 + v(t)N_0$, then $\gamma_0+v(t)N_0=\gamma(t)+uN(t)$\\
\[v(t)=\frac{\left|\begin{array}{cc}\ &\ \\ \gamma(t)-\gamma_0&N(t)\\\ &\ \end{array}\right|}{\left|\begin{array}{cc}\ &\ \\ N_0& N(t)\\\ &\ \end{array}\right|}\]\\
Assume that $t$ is arc-length parameter, then:\\
\begin{align}
\lim_{t\to 0} P(t) &= \lim_{t\to 0}\left(\gamma_0 + v(t)N_0\right)\\
&= \gamma_0 + \lim_{t\to 0}v(t)N_0\\
&= \gamma_0 + \lim_{t\to 0}\frac{\left|\begin{array}{cc}\ &\ \\ \gamma(t)-\gamma_0&N(t)\\\ &\ \end{array}\right|}{\left|\begin{array}{cc}\ &\ \\ N_0& N(t)\\\ &\ \end{array}\right|} N_0\\
&= \gamma_0 + \lim_{t\to 0}\frac{\left(\left|\begin{array}{cc}\ &\ \\ \gamma(t)-\gamma_0&N(t)\\\ &\ \end{array}\right|\right)'}{\left(\left|\begin{array}{cc}\ &\ \\ N_0& N(t)\\\ &\ \end{array}\right|\right)'} N_0\\
&= \gamma_0 + \lim_{t\to 0}\frac{\left|\begin{array}{cc}\ &\ \\ \gamma'(t)&N(t)\\\ &\ \end{array}\right|+\left|\begin{array}{cc}\ &\ \\ \gamma(t)-\gamma_0&N'(t)\\\ &\ \end{array}\right|}{\left|\begin{array}{cc}\ &\ \\ N_0& N'(t)\\\ &\ \end{array}\right|} N_0\\
&= \gamma_0 + \frac{\left|\begin{array}{cc}\ &\ \\ T_0&N_0\\\ &\ \end{array}\right|}{\left|\begin{array}{cc}\ &\ \\ N_0& -\kappa_0T_0\\\ &\ \end{array}\right|} N_0\\
&= \gamma_0 + \frac{1}{\kappa_0} N_0
\end{align}
\end{proof}
\end{document}
