\documentclass[10pt,a4paper]{article}
\usepackage{amsmath, amssymb, amsthm}
%加這個就可以設定字體
\usepackage{fontspec}
\usepackage{xkeyval} %MikTeX 2.9 版本相容有誤, 以此修正
%使用xeCJK,其他的還有CJK或是xCJK
\usepackage{xeCJK}
%設定英文字型,不設的話就會使用預設的字型
%\setmainfont{Times New Roman}
%設定中英文的字型
%字型的設定可以使用系統內的字型,而不用像以前一樣另外安裝
%\setCJKmainfont{文泉驛微米黑}
\setCJKmainfont{WenQuanYi Micro Hei}
\setCJKfamilyfont{lh}{LiHei Pro}
\newcommand{\LiHei}{\CJKfamily{lh}}
%中文自動換行
\XeTeXlinebreaklocale "zh"
%文字的彈性間距
\XeTeXlinebreakskip = 0pt plus 1pt
%設定段落之間的距離
\setlength{\parskip}{0.3cm}
%設定行距
%\linespread{1.5}\selectfont
\usepackage{enumerate}

%先暫時用水泥字型,如果任何人看不下去就改吧XD
\usepackage[T1]{fontenc}
\usepackage{concmath}
%\usepackage{mathptmx} %Times

\usepackage{tabularx}

\newcounter{theProblemCounter}
\newtheorem{problem}[theProblemCounter]{Problem}

\begin{document}
\title{{\fontspec{Copperplate Gothic Bold}Geometry Homework 11}}
%\title{{\fontspec{Copperplate}Geometry Homework 11}}
\author{{\it{B96201044}} {\LiHei 黃上恩}, {\it{B98901182}} {\LiHei 時丕勳}, {\it{K0020100x}} {\LiHei 劉士瑋}}
\date{\today}
\maketitle

\newcommand{\bx}{\mathbb{X}}
\newcommand{\bfx}{\mathbf{x}}
\newcommand{\grad}{\textrm{grad }}
\newcommand{\sech}{\mbox{sech}}

%第四題
\setcounter{theProblemCounter}{3}
\begin{problem}[Ex p261 8.]
Show that if all the geodesics of a connected surface are plane curves, then the surface is contained in a plane or a sphere.
\end{problem}
\begin{proof}

\end{proof}

%第五題
\setcounter{theProblemCounter}{4}
\begin{problem}[Ex p262 17.]
Let $\alpha: I\to S$ be a curve parametrized by arc length $s$, with nonzero curvature. Consider the parametrized surface
\[ \bfx(s, v) = \alpha(s)+vb(s), \ \ s\in I, -\epsilon<v<\epsilon, \epsilon > 0,\]
where $b$ is the binormal vector of $\alpha$. Prove that if $\epsilon$ is small, $\bfx(I\times (-\epsilon, \epsilon)) = S$ is a regular surface over which $\alpha(I)$ is geodesic. (thus, every curve is a geodesic on the surface generated by its binormals).
\end{problem}
\begin{proof}
\begin{align*}
\bfx_s&=\alpha'(s)+vb'(s)\\
&=t(s)+v\tau(s)n(s)\\
\bfx_v&=b(s)\\
\rightarrow \bfx_s\times\bfx_v&=-n(s)+v\tau(s)t(s)\\
&\neq 0
\end{align*}
So $\bfx$ is a regualr surface.\\
Since $\alpha''(s)=n(s)$, and at $v=0$, $N\parallel \bfx_s\times\bfx_v=-n(s)$. So $\alpha'(s)\parallel N$, and $\kappa_g=0$. So $\alpha(I)$ is geodesic.
\end{proof}

%第八題
\setcounter{theProblemCounter}{7}
\begin{problem}
用 (A) 表示在座標變換下不變、用 (B) 表示在 isometry 下不變 (保 $E, F, G$) 下的性質

\begin{tabular}{c|c|c|c|c|c|c|}
\cline{2-7}
 & line of curvature & geodesic & asymptotic curve & $\Gamma^{k}_{ij}$ & $H$ & $K$ \\
\cline{2-7}
(A) & Yes$_{(1)}$ & Yes$_{(1)}$ & Yes$_{(1)}$ & No$_{(2)}$ & Yes$_{(1)}$ & Yes$_{(1)}$\\
\cline{2-7}
(B) & No$_{(6)}$ & Yes$_{\textrm{problem 9(a)}}$ & No$_{(6)}$ & Yes$_{(3)}$ & No$_{(5)}$ & Yes$_{(4)}$\\
\cline{2-7}
\end{tabular}
\end{problem}
\begin{proof}
\begin{enumerate}
\item[(1)]
Since curves, surface, T,A,N, t,n,b are all geometry objects, $\kappa_n$, $\kappa_g$, $\tau_g$ are geometry objects too. So line of curvature, geodesic, asymptotic curve are also geometry objects. Since principal direction and principal curvature are geometry objects too, $H$ and $K$ are geometry objects.
\item[(2)]
Consider an surface $\bx(u,v)$ and $\hat{\bx}(u,v)=\bx(v,u)$, it's trivial that $\hat{\Gamma}^2_{11}=\Gamma^1_{22}$, so $\hat{\Gamma}^2_{11}\neq\Gamma^2_{11}$ when $\Gamma^2_{11}\neq \Gamma^1_{22}$, and it's trivial to find a surface with $\Gamma^2_{11}\neq \Gamma^1_{22}$ (For example, $\bx(u,v)=(u\cos{v},u\sin{v},0)$. As shown in HW 10, $\Gamma_{22}^1=-u\neq \Gamma{11}^2=0$).
\item[(3)]
Since $\Gamma_{ij}^k=g^{kl}[i,j,l]$, and both $g^{kl}$ and $[i,j,l]$ only depends on $g_{ij}$, $\Gamma_{ij}^k$ is same in isometry.
\item[(4)]
Gauss Theorema Egregium.
\item[(5)]
Consider a plane $(u,v,0)$ and a cone $(u,\cos{v},\sin{v})$. They are isometric but have different $H$.
\item[(6)]
Consider a plane $(u,v,0)$ and a cone $(u,\cos{v},\sin{v})$. Since every line in the plane is line of curvature, but there are only two line of curvatures passing one point in cone, some line of curvatures in the plane is not a line of curvature in cone. $u=0$ is asymptotic curve of the plane, but it's not an asymtotic curve of the cone.
\end{enumerate}
\end{proof}

%第九題
\setcounter{theProblemCounter}{8}
\begin{problem}
考慮 p221, p222 中 helicoid $Y$ 和 catenoid $X$ 的 parametrization。\\
$X(u,v)=(a\cosh{v}\cos{u},a\cosh{v}\sin{u},av)$, $Y(u,v)=(a\sinh{v}\cos{u},a\sinh{v}\sin{u},au)$
\begin{enumerate}
\item[(a)] $X$ 中的 geodesics 相對應映到 $Y$ 中也是 geodesics 嗎?
\item[(b)] 已知 $X$ 的經線 ($u = $ const) 與 $v=0$ 都是 geodesics。描述他們在 $Y$ 中的對應曲線?他們都是 geodesics 嗎?
\end{enumerate}
\end{problem}
\begin{proof}
\begin{enumerate}
\item[(a)]
Since $X$ and $Y$ are isometry, they have the same $\Gamma^{k}_{ij}$, and have the same geodesic equation. So geodesic in $X$ is also geodesic in $Y$.
\item[(b)]
when $u= $ const, $Y(u,v)$ is a line $(a\cos{C}\sinh{v}, a\sin{C}\sinh{v},aC)$, and is a geodesic of $Y$.\\
when $v=0$, $Y(u,v)$ is a line $(0,0,au)$, and is a geodesic of $Y$.
\end{enumerate}
\end{proof}
\end{document}
