\documentclass[10pt,a4paper]{article}
\usepackage{amsmath, amssymb, amsthm}
%加這個就可以設定字體
\usepackage{fontspec}
\usepackage{xkeyval} %MikTeX 2.9 版本相容有誤, 以此修正
%使用xeCJK,其他的還有CJK或是xCJK
\usepackage{xeCJK}
%設定英文字型,不設的話就會使用預設的字型
%\setmainfont{Times New Roman}
%設定中英文的字型
%字型的設定可以使用系統內的字型,而不用像以前一樣另外安裝
%\setCJKmainfont{文泉驛微米黑}
\setCJKmainfont{WenQuanYi Micro Hei}
\setCJKfamilyfont{lh}{LiHei Pro}
\newcommand{\LiHei}{\CJKfamily{lh}}
%中文自動換行
\XeTeXlinebreaklocale "zh"
%文字的彈性間距
\XeTeXlinebreakskip = 0pt plus 1pt
%設定段落之間的距離
\setlength{\parskip}{0.3cm}
%設定行距
%\linespread{1.5}\selectfont
\usepackage{enumerate}

%先暫時用水泥字型,如果任何人看不下去就改吧XD
\usepackage[T1]{fontenc}
\usepackage{concmath}
%\usepackage{mathptmx} %Times

\usepackage{tabularx}

\newcounter{theProblemCounter}
\newtheorem{problem}[theProblemCounter]{Problem}

\begin{document}
%\title{{\fontspec{Copperplate Gothic Bold}Geometry Homework 12}}
\title{{\fontspec{Copperplate}Geometry Homework 12}}
\author{{\it{B96201044}} {\LiHei 黃上恩}, {\it{B98901182}} {\LiHei 時丕勳}, {\it{K0020100x}} {\LiHei 劉士瑋}}
\date{\today}
\maketitle

\newcommand{\bx}{\mathbb{X}}
\newcommand{\bfx}{\mathbf{x}}
\newcommand{\grad}{\textrm{grad }}
\newcommand{\sech}{\mbox{sech}}
\newcommand{\pr}[2]{\frac{\partial #1}{\partial #2}}
\newcommand{\prr}[3]{\frac{\partial^2 #1}{\partial #2\partial #3}}
\newcommand{\ip}[2]{\left\langle#1, #2\right\rangle}

%第三題
\setcounter{theProblemCounter}{2}
\begin{problem}[Ex p294 3.]
If $p$ is a point of a regular surface $S$, prove that \[ K(p) = \lim_{r\to 0}\frac{12}{\pi}\frac{\pi r^2-A}{r^4}, \] where $K(p)$ is the Gaussian curvature of $S$ at $p$, $r$ is the radius of a geodesic circle $S_r(p)$ centered in $p$, and $A$ is the area of the region bounded by $S_r(p)$.
\end{problem}
\begin{proof}
\begin{align*}
A_R&=\int_{0}^{R}\int_0^{2\pi}\sqrt{EG-F^2}d\theta dr\\
&=\int_{0}^{R}\int_0^{2\pi}\sqrt{G}d\theta dr\\
&\approx \int_{0}^{R}\int_0^{2\pi} r-\frac{K}{6}r^3 d\theta dr\\
&=\int_0^{2\pi}\frac{1}{2}R^2-\frac{K}{24}R^4 d\theta\\
&=\pi R^2-\frac{R^4}{24}\int_0^{2\pi}K d\theta\\
\rightarrow \frac{1}{2\pi}\int_0^{2\pi}K d\theta &=\frac{12}{r^4}(r^2-\frac{1}{\pi}A_r)\\
\rightarrow K(p)&=\lim_{r\to 0}\frac{12}{r^4}(r^2-\frac{1}{\pi}A_r)\\
&=\lim_{r\to 0}\frac{12}{\pi}\frac{\pi r^2-A_r}{r^4}\\
\end{align*}
\end{proof}

%第四題
\setcounter{theProblemCounter}{3}
\begin{problem}[Ex p295 4.]
Show that in a system of normal coordinates centered in $p$, all the Christoffel symbols are zero at $p$.
\end{problem}

%第五題
\setcounter{theProblemCounter}{4}
\begin{problem}[Ex p295 5.]
For which of the pair of surfaces given below does there exist a local isometry?
\begin{enumerate}
\item[(a)] Torus of revolution and cone.
\item[(b)] Cone and sphere.
\item[(c)] Cone and cylinder.
\end{enumerate}
\end{problem}

%第八題
\setcounter{theProblemCounter}{7}
\begin{problem}\hspace*{1em}
\begin{enumerate}
\item[(a)] 在半徑 $R$ 的球面上,計算 geodesic circle 的長度,並驗證 P292 課文中間 $K(p)$ 的公式。
\item[(b)] 用一樣的精神,檢驗 P294 3. 的公式。
\end{enumerate}
\end{problem}
\begin{proof}
\begin{enumerate}
\item[(a)]
WLOG, let $p = (0, 0, R)$, If $q\in T_p$ with $q = (l, \theta)$, then
$$
\exp(q) = \left(R\sin\frac{l}{R}\cos\theta, R\sin\frac{l}{R}\sin\theta, R\cos\frac{l}{R}\right),
$$
and thus the length of the image of the circle $\{q\in T_p : d(q, p) = l\}$ is
$$
2\pi\ip{\pr{\exp(q)}{\theta}}{\pr{\exp(q)}{\theta}}^{1/2} = 2\pi\left|R\sin\frac{l}{R}\right|
$$
. When $l\rightarrow 0$, it is $$2\pi R\sin\frac{l}{R}.$$

, which is the length of the geodesic circle. By the formula,
\begin{align*}
K(p) &= \lim_{r\rightarrow 0}\frac{3}{\pi}\frac{2\pi r - L}{r^3} = \lim_{r\rightarrow 0}\frac{3}{\pi}\frac{2\pi r - 2\pi R\sin\frac{l}{R}}{r^3}\\ &\approx \lim_{r\rightarrow 0}\frac{3}{\pi}\frac{2\pi r - 2\pi R\left(r/R - \left(\frac rR\right)^3/6\right)}{r^3} \\ &=
\lim_{r\rightarrow 0}\frac{3}{\pi}\frac{2\pi r\left(\frac rR\right)^3/6}{r^3} = \frac{1}{R^2}.
\end{align*}

\item[(b)]
The area bounded by the geodesic circle is
$$
2\pi R\int^l_0 \left|\sin\frac{r}{R}\right| dr.
$$
When $l\rightarrow 0$, it is $$2\pi R^2 - 2\pi R^2\cos\frac lR.$$

By the formula, 
\begin{align*}
K(p) &= \lim_{r\rightarrow 0}\frac{12}{\pi}\frac{\pi r^2 - A}{r^4} = \lim_{r\rightarrow 0}\frac{12}{\pi}\frac{\pi r^2 + 2\pi R^2\cos\frac rR - 2\pi R^2}{r^4}\\ &\approx \lim_{r\rightarrow 0}\frac{12}{\pi}\frac{\pi r^2 + 2\pi R^2\left(1 - \left(\frac rR\right)^2/2 + \left(\frac rR\right)^4/24\right) - 2\pi R^2}{r^4} \\ &=
\lim_{r\rightarrow 0}\frac{12}{\pi}\frac{\pi R^2\left(\frac rR\right)^4/12}{r^4} = \frac{1}{R^2}
\end{align*}
\end{enumerate}
\end{proof}
\end{document}
