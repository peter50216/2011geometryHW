\documentclass[10pt,a4paper]{article}
\usepackage{amsmath, amssymb, amsthm}
%加這個就可以設定字體
\usepackage{fontspec}
\usepackage{xkeyval} %MikTeX 2.9 版本相容有誤, 以此修正
%使用xeCJK,其他的還有CJK或是xCJK
\usepackage{xeCJK}
%設定英文字型,不設的話就會使用預設的字型
%\setmainfont{Times New Roman}
%設定中英文的字型
%字型的設定可以使用系統內的字型,而不用像以前一樣另外安裝
%\setCJKmainfont{文泉驛微米黑}
\setCJKmainfont{WenQuanYi Micro Hei}
\setCJKfamilyfont{lh}{LiHei Pro}
\newcommand{\LiHei}{\CJKfamily{lh}}
%中文自動換行
\XeTeXlinebreaklocale "zh"
%文字的彈性間距
\XeTeXlinebreakskip = 0pt plus 1pt
%設定段落之間的距離
\setlength{\parskip}{0.3cm}
%設定行距
%\linespread{1.5}\selectfont
\usepackage{enumerate}

%先暫時用水泥字型,如果任何人看不下去就改吧XD
\usepackage[T1]{fontenc}
\usepackage{concmath}
%\usepackage{mathptmx} %Times

\newcounter{theProblemCounter}
\newtheorem{problem}[theProblemCounter]{Problem}

\begin{document}
%\title{{\fontspec{Copperplate Gothic Bold}Geometry Homework 8}}
\title{{\fontspec{Copperplate}Geometry Homework 8}}
\author{{\it{B96201044}} {\LiHei 黃上恩}, {\it{B98901182}} {\LiHei 時丕勳}, {\it{K0020100x}} {\LiHei 劉士瑋}}% 電波好皮呦~
\date{\today}
\maketitle

\newcommand{\bx}{\mathbb{X}}
\newcommand{\bfx}{\mathbf{X}}
\newcommand{\sech}{\mbox{sech}}
%\newcommand{\cosh}{\mbox{cosh}\ }
%\newcommand{\tanh}{\mbox{tanh}\ }
%\newcommand{\sinh}{\mbox{sinh}\ }
%第二題
\setcounter{theProblemCounter}{1}
\begin{problem}
考慮直線族 $L_\lambda: \frac{x}{\lambda} + \frac{y}{1-\lambda}=1$,令 ruled surface $\bx$ 為 $(L_\lambda, \lambda)\subset \mathbb{R}^2\times \mathbb{R}$
\begin{enumerate}
\item[(a)] 求出 line of striction(龍骨) $\beta(\lambda)\in\mathbb{R}^3$
\item[(b)] 令 $\gamma(\lambda)$ 為 $\beta(\lambda)$ 在 $\mathbb{R}^2$ 上的投影,說明 $L_\lambda$ 為 $\gamma(\lambda)$ 的切線
\item[(c)] $\gamma(\lambda)$ 是圓嗎?其方程式為何(以 $f(x,y)=c$ 的方式表示)?
\end{enumerate}
\end{problem}
\begin{proof}
\end{proof}

%第四題
\setcounter{theProblemCounter}{3}
\begin{problem}[Ex p.210 6]
Let \[
\bfx(t, v)=\alpha(t)+vw(t)
\]
be a developable surface. Prove that at a regular point we have
\[
\left\langle N_v, \bfx_v\right\rangle
=\left\langle N_v, \bfx_t\right\rangle=0.\]
Conclude that \emph{the tangent plane of a developable surface is constant along} (the regular points of) \emph{a fixed ruling}.
\end{problem}
\begin{proof}
\begin{align*}
\bfx_{vv}=&0 \Rightarrow g=\langle N, \bfx_{vv}\rangle = 0; \\
K=&\det(-dN)=0 \Rightarrow eg=f^2 \Rightarrow f = 0; \\
N_v =& dN(\bfx_v) = \left[\begin{array}{cc}E&F\\F&G\end{array}\right]^{-1}\left[\begin{array}{cc}e&0\\0&0\end{array}\right]\left[\begin{array}{c}0\\1\end{array}\right]=\left[\begin{array}{c}0\\0\end{array}\right]; \\
\Rightarrow \langle N_v, \bfx_v\rangle =& \langle N_v, \bfx_t\rangle = 0.
\end{align*}
Thus $N$, the normal vector of the tangent plane, is independent of $v$ and hence the conclusion follows.
\end{proof}

%第五題
\setcounter{theProblemCounter}{4}
\begin{problem}[Ex p.210 8]
Show that if $C\subset S^2$ is a parallel of a unit sphere $S^2$, then the envelope of tangent planes of $S^2$ along $C$ is either a cylinder, if $C$ is an equator, or a cone, if $C$ is not an equator.
\end{problem}
\begin{proof}
WLOG, let the unit sphere's centre be the origin and let the plane on which the $C$ is be parallel to the xy-plane. If $C$ is an equator, that is, on the xy-plane, the tangent plane of each point is therefore parallel to the z-axis and thus the envelope form a cylinder. Hence consider that $C$ is not on the $xy$ plane.  By the symmetry of $S$ and $C$, the intersection of the envelope and any plane containing z-axis is identical up to rotation along z-axis. Picking such a plane and observing that the intersection being a line should intersect z-axis at exactly one point since $\alpha \neq 0$, we conclude that each intersection passes through the very point in z-axis. Let the point in z-axis be the generator of the envelope. Since each ruler should pass through exactly one point in $C$, the envelope therefore forms a cone.
\begin{align*}
\end{align*}
\end{proof}
\end{document}
