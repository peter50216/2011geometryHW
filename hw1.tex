\documentclass[12pt,a4paper]{article}
%加這個就可以設定字體
\usepackage{fontspec}
\usepackage{xkeyval} %MikTeX 2.9 版本相容有誤, 以此修正
%使用xeCJK,其他的還有CJK或是xCJK
\usepackage{xeCJK}
%設定英文字型,不設的話就會使用預設的字型
%\setmainfont{Times New Roman}
%設定中英文的字型
%字型的設定可以使用系統內的字型,而不用像以前一樣另外安裝
\setCJKmainfont{LiHei Pro}
%中文自動換行
\XeTeXlinebreaklocale "zh"
%文字的彈性間距
\XeTeXlinebreakskip = 0pt plus 1pt
%設定段落之間的距離
\setlength{\parskip}{0.3cm}
%設定行距
%\linespread{1.5}\selectfont

%先暫時用水泥字型,如果任何人看不下去就改吧XD
\usepackage[T1]{fontenc}
\usepackage{concmath}

\usepackage{amsmath, amssymb, amsthm}
\newtheorem{problem}{Problem}


\begin{document}
\title{Geometry Homework 1}
\author{{\it{B96201044}} 黃上恩, {\it{B98901182}} 時丕勳, {\it{K0020100x}} 劉士瑋}
\date{\today}
\maketitle

\begin{problem}[P7: 4]
Let $\alpha:(0, \pi)\to \mathbf{R}^2$ be given by
\[ \alpha(t) = \left(\cos t, \cos t + \log\tan\frac{t}{2}\right),\]
where $t$ is the angle that the $y$ axis makes with the vector $\alpha(t)$. The trace of $\alpha$ is called the \emph{tractrix} (Fig. 1-9). Show that
\begin{enumerate}
\item[(a)] $\alpha$ is a differentiable parametrized curve, regular except at $t=\pi/2$.
\item[(b)] The length of the segment of the tangent of the tractrix between the point of tangency and the $y$ axis is constantly equal to $1$.
\end{enumerate}
\end{problem}

\begin{proof}
This is the proof.
\end{proof}

\begin{problem}[Curvature is a geometric object I.]
haha
\end{problem}

\end{document}