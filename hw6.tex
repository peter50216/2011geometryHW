\documentclass[10pt,a4paper]{article}
\usepackage{amsmath, amssymb, amsthm}
%加這個就可以設定字體
\usepackage{fontspec}
\usepackage{xkeyval} %MikTeX 2.9 版本相容有誤, 以此修正
%使用xeCJK,其他的還有CJK或是xCJK
\usepackage{xeCJK}
%設定英文字型,不設的話就會使用預設的字型
%\setmainfont{Times New Roman}
%設定中英文的字型
%字型的設定可以使用系統內的字型,而不用像以前一樣另外安裝
%\setCJKmainfont{文泉驛微米黑}
\setCJKmainfont{WenQuanYi Micro Hei}
\setCJKfamilyfont{lh}{LiHei Pro}
\newcommand{\LiHei}{\CJKfamily{lh}}
%中文自動換行
\XeTeXlinebreaklocale "zh"
%文字的彈性間距
\XeTeXlinebreakskip = 0pt plus 1pt
%設定段落之間的距離
\setlength{\parskip}{0.3cm}
%設定行距
%\linespread{1.5}\selectfont

%先暫時用水泥字型,如果任何人看不下去就改吧XD
\usepackage[T1]{fontenc}
\usepackage{concmath}
%\usepackage{mathptmx} %Times

\newcounter{theProblemCounter}
\newtheorem{problem}[theProblemCounter]{Problem}

\begin{document}
\title{{\fontspec{Copperplate Gothic Bold}Geometry Homework 6}}
%\title{{\fontspec{Copperplate}Geometry Homework 5}}
\author{{\it{B96201044}} {\LiHei 黃上恩}, {\it{B98901182}} {\LiHei 時丕勳}, {\it{K0020100x}} {\LiHei 劉士瑋}}
\date{\today}
\maketitle

\newcommand{\bx}{\mathbb{X}}
%第一題
\setcounter{theProblemCounter}{0}
\begin{problem}[Ex P151 2]
Show that if a surface is tangent to a plane along a curve, then the points of this curve are either parabolic or planar.
\end{problem}
\begin{proof}
\end{proof}

%第三題
\setcounter{theProblemCounter}{2}
\begin{problem}[Ex P151 3]
\begin{enumerate}
\item[]
\item[(a)] Let $C\subset S$ be a regular curve on a surface $S$ with Gaussian curvature $K > 0$. Show that the curvature $\kappa$ of $C$ at $p$ satisfies \[ \kappa\ge \min(|\kappa_1|, |\kappa_2|),\] where $\kappa_1, \kappa_2$ are the principal curvatures of $S$ at $p$.
\item[(b)] 為什麼上一小題需要 $K>0$ 的條件,$K\ge 0$ 不可以嗎?
\end{enumerate}
\end{problem}

%第七題
\setcounter{theProblemCounter}{6}
\begin{problem}
\begin{enumerate}
\item[]
\item[(a)] $T_\lambda$ 是縮放 $\lambda$ 倍的映射,$\lambda>0$。$\mathbb{X}:\Omega\to \mathbb{R}^3$ regular surface。討論 $T_\lambda\circ \mathbb{X}:\Omega\to\mathbb{R}^3$ 上對應點 $\kappa_n, H, K$ 的變化。
\item[(b)] $\mathbb{X}:\begin{array}{c}\Omega\\(u,v)\end{array}\to \mathbb{R}^3$,若定義 $\overline{\mathbb{X}}(u, v)= \mathbb{X}(v, u)$(因此 $N$ 轉向)。討論 $\overline{\mathbb{X}}(\Omega)$ 上相對應點的 $K_n, H, K$ 變化。
\end{enumerate}
\end{problem}

%第九題
\setcounter{theProblemCounter}{8}
\begin{problem}[旋轉面]
$\mathbb{X}(u, v)=(f(u)\cos v, f(u)\sin v, g(u))$,$f>0$ \begin{enumerate}
\item[(a)] 計算其 $e, f, g, H, K$
\item[(b)] 討論其 principal direction 與 principal curvature $K_1, K_2$。
\end{enumerate}
\end{problem}

%第十題
\setcounter{theProblemCounter}{9}
\begin{problem}[管面]
$\mathbb{X}(s,\theta) = \gamma(s)+\cos\theta\vec{n}(s) + \sin\theta\vec{b}(s)$, $0<\kappa < 1$
\begin{enumerate}
\item[(a)] 計算其 $e, f, g, H, K$
\item[(b)] 討論曲面上 $K$ 的分佈。
\end{enumerate}
\end{problem}
\end{document}
